\documentclass{llncs}
\usepackage{graphicx}

\begin{document}

\title{Probabilistic Reactive Obstacle Avoidance}
\author{Pete Florence\inst{1} \and John Carter\inst{1} \and Russ Tedrake\inst{1} }
\institute{MIT Computer Science and Artificial Intelligence Laboratory, Cambridge, MA \email{{peteflo,jcarter,russt}@csail.mit.edu}}



\maketitle

\begin{abstract}
The abstract text goes here.
\end{abstract}

\section{Introduction}
Here is the text of your introduction.

\subsection{And I can have second-order titles too}

\begin{equation}
    \label{simple_equation}
    \alpha = \sqrt{ \beta }
\end{equation}

\subsection{Subsection Heading Here}
Write your subsection text here.

\begin{figure}
    \centering
    \includegraphics[width=3.0in]{images/baymax.png}
    \caption{Simulation Results}
    \label{simulationfigure}
\end{figure}



\section{Conclusion}
Write your conclusion here.


\section{Conclusion}
Write your conclusion here.


\section{Fixed-Period Problems: The Sublinear Case}
%
With this chapter, the preliminaries are over, and we begin the
search for periodic solutions \dots
%
\subsection{Autonomous Systems}
%
In this section we will consider the case when the Hamiltonian
$H(x)$ \dots
%
\subsubsection*{The General Case: Nontriviality.}
%
We assume that $H$ is
$\left(A_{\infty}, B_{\infty}\right)$-subqua\-dra\-tic
at infinity, for some constant \dots
%
\paragraph{Notes and Comments.}
The first results on subharmonics were \dots
%
\begin{proposition}
Assume $H?(0)=0$ and $ H(0)=0$. Set \dots
\end{proposition}
\begin{proof}[of proposition]
Condition (8) means that, for every $\delta?>\delta$, there is
some $\varepsilon>0$ such that \dots \qed
\end{proof}
%
\begin{example}[\rmfamily (External forcing)]
Consider the system \dots
\end{example}
\begin{corollary}
Assume $H$ is $C^{2}$ and
$\left(a_{\infty}, b_{\infty}\right)$-subquadratic
at infinity. Let \dots
\end{corollary}
\begin{lemma}
Assume that $H$ is $C^{2}$ on $\bbbr^{2n}\backslash \{0\}$
and that $H??(x)$ is \dots
\end{lemma}
\begin{theorem}[(Ghoussoub-Preiss)]
Let $X$ be a Banach Space and $\Phi:X\to\bbbr$ \dots
14 LATEX2? Class for Lecture Notes in Computer Science
\end{theorem}
\begin{definition}
We shall say that a $C^{1}$ function $\Phi:X\to\bbbr$
satisfies \dots
\end{definition}

\end{document}